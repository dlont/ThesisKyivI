%%%%%%%%%%%%%%%%%%%%%%%%%%%%%%%ET%%%%%%%%%%%%%%%%%%%%%%%%%%%
\begin{figure}[p!]
\begin{center}
\begin{subfloat}[]{\includegraphics[width=0.45\textwidth] {figures/dijets_php_2012/figures/scalingdij07etppdfcol}
   \label{fig:csetbar_subfig1}
 }%
\end{subfloat}
 \begin{subfloat}[]{\includegraphics[width=0.45\textwidth]{figures/dijets_php_2012/figures/scalingdij07etgpdfcol}
   \label{fig:csetbar_subfig2}
 }%
\end{subfloat}
\caption{Поперечний переріз утворення подій з двома адронними струменями як функція середньої поперечної енергії пари найбільш енергетичних струменів.}
\label{fig:csetbar}
\end{center}
\end{figure}

%%%%%%%%%%%%%%%%%%%%%%%%%%%%%%%%%ETA%%%%%%%%%%%%%%%%%%%%%
\begin{figure}[p!]
\begin{center}
\begin{subfloat}[]{\includegraphics[width=0.45\textwidth] {figures/dijets_php_2012/figures/scalingdij07etappdfcol}
   \label{fig:csetabar_subfig1}
 }%
\end{subfloat}
 \begin{subfloat}[]{\includegraphics[width=0.45\textwidth]{figures/dijets_php_2012/figures/scalingdij07etagpdfcol}
   \label{fig:csetabar_subfig2}
 }%
\end{subfloat}
\caption{Поперечний переріз утворення подій з двома адронними струменями як функція середньої псевдошвидкісності пари найбільш енергетичних струменів.}
\label{fig:csetabar}
\end{center}
\end{figure}

%%%%%%%%%%%%%%%%%%%%%%%%%%%%%MJJ%%%%%%%%%%%%%%%%%%%%%%%%
\begin{figure}[p!]
\begin{center}
\begin{subfloat}[]{\includegraphics[width=0.45\textwidth] {figures/dijets_php_2012/figures/scalingdij07mjjppdfcol}
   \label{fig:csmjj_subfig1}
 }%
\end{subfloat}
 \begin{subfloat}[]{\includegraphics[width=0.45\textwidth]{figures/dijets_php_2012/figures/scalingdij07mjjgpdfcol}
   \label{fig:csmjj_subfig2}
 }%
\end{subfloat}
\caption{Поперечний переріз утворення подій з двома адронними струменями як функція інваріантної маси пари найбільш енергетичних струменів.}
\label{fig:csmjj}
\end{center}
\end{figure}

%%%%%%%%%%%%%%%%%%%%%%%%%%%%%COS THETA%%%%%%%%%%%%%%%%%%%%%%%%
\begin{figure}[p!]
\begin{center}
\begin{subfloat}[]{\includegraphics[width=0.45\textwidth] {figures/dijets_php_2012/figures/scalingdij07costppdfcol}
   \label{fig:csetabar_subfig1}
 }%
\end{subfloat}
 \begin{subfloat}[]{\includegraphics[width=0.45\textwidth]{figures/dijets_php_2012/figures/scalingdij07costgpdfcol}
   \label{fig:csetabar_subfig2}
 }%
\end{subfloat}
\caption{Поперечний переріз утворення подій з двома струменями як функція кута розсіяння у системі центру мас $\gamma^\ast p$.}
\label{fig:csetabar}
\end{center}
\end{figure}

%%%%%%%%%%%%%%%%%%%%%%%%%%%%%%%X GAMMA%%%%%%%%%%%%%%%%%%%%%
\begin{figure}[p!]
\begin{center}
\begin{subfloat}[]{\includegraphics[width=0.45\textwidth] {figures/dijets_php_2012/figures/scalingdij07xgppdfcol}
   \label{fig:csxg_subfig1}
 }%
\end{subfloat}
 \begin{subfloat}[]{\includegraphics[width=0.45\textwidth]{figures/dijets_php_2012/figures/scalingdij07xggpdfcol}
   \label{fig:csxg_subfig2}
 }%
\end{subfloat}
\caption{Поперечний переріз утворення подій з двома адронними струменями як функція частки поздовжньої компоненти імпульсу фотона.}
\label{fig:csxg}
\end{center}
\end{figure}