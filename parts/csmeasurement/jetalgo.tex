Як було описано у Розділі~\ref{ch:theory}, партони не існують у вільному стані, а утворюють зв'язані системи --- адрони. У експерименті партони проявляються як сукупність частинок, що мають близькі за напрямком імпульси, та утворюють колімовані пучки --- адронні струмені. Дочірні частинки несуть інформацію про величину та початковий напрямок 4-імпульсу материнського партона, тому, вимірюючи кінематичні характеристики цих частинок, можливо відновити деталі жорсткої реакції (4-імпульси партонів, що приймають участь у реакції). Умовно цю проблему можна поділити на два етапи:
\begin{itemize}
	\item через те, що продуктами жорсткої реакції можуть бути декілька партонів, на першому етапі необхідно ідентифікувати приналежність адронів кінцевого стану до певного партона;
	\item на другому етапі необхідно реконструювати кінематичні змінні партонів через 4-імпульси дочірніх частинок.
\end{itemize}
На практиці ці задачі вирішуються шляхом застосування алгоритму реконструкції адронних струменів. Будь-який алгоритм має задовольняти наступним вимогам:
\begin{itemize}
	\item бути незалежним по відношенню до м'якого та колінеарного випромінювання (див. Главу~\ref{sec:realcorrections});
	\item зберігати властивість факторизації теоретичних передбачені у рамках КХД;
	\item мала чутливість по відношенню до ефектів адронізації та розпаду дочірніх частинок;
	\item низька чутливість до ефектів пов'язаних зі взаємодією з уламками початкового протона;
	\item інваріантність по відношенню до поздовжніх Лоренц-перетворень системи координат;
	\item ефективність реалізації у вигляді програмного коду\footnote{Типово, реконструкція адронних струменів є самою копіткою задачею і потребує найбільше машинного часу.}.
\end{itemize}
На даний момент існую декілька алгоритмів, що задовольняють усім умовам. Детальний опис цих алгоритмів можна знайти в~\cite{salam:antikt}. У роботі~\cite{claudia:antikt} було продемонстровано, що так званий, поздовжньо-інваріантний \kt-кластерний алгоритм~\cite{ktcatani}, має оптимальні характеристики для реконструкції адронних струменів у $\ep$ зіткненнях, які вивчалися у даній роботі. Цей алгоритм вперше було використано для $e^{+}e^{-}$ зіткнень~\cite{ktepcollisions}, але згодом він був адаптований для лептон-адронних взаємодій~\cite{ktclaudiaradiusdependence}. 

Як вхідні дані для алгоритму використовувалися енергії та імпульси, що були виміряні у комірках калориметра, у симульованих подіях цей алгоритм застосовувався також до партонів та адронів. Алгоритм знаходить відстань між кожною частинкою та прямою, що відповідає напрямку протонного пучка в імпульсному просторі,
\begin{equation}
	d_{i} = E_{T,i}^{2},	
\end{equation}
а також відстань між двома об'єктами за формулою:
\begin{equation}
	d_{ij} = \min\left(E_{T,i}^{2}, E_{T,j}^{2}\right)\left[\left(\eta_{i} - \eta_{j}\right)^{2} + \left(\phi_{i} - \phi_{j}\right)^{2}\right].
\end{equation}
Відстані $d_{i}$ та $d_{ij}$ обраховуються для всіх об'єктів. Далі знаходиться мінімальна величина з усіх $d_і$ та $d_{ij}$. Якщо мінімальною є одна за $d_{ij}$, тоді пара об'єктів комбінується у один за обраною схемою. Якщо мінімальною виявляється $d_{i}$, тоді цей об'єкт визначається як адронний струмінь і виключається з алгоритму. У даному дослідженні була використана наступна схема комбінації об'єктів:
\begin{equation}
	E_{t,k}^{2} = E_{T,і}^{2} + E_{T,j}^{2};
\end{equation}
\begin{equation}
	\eta_{k} = \frac{E_{T,і}\eta_{i} + E_{T,j}\eta_{j}}{E_{T,і} + E_{T,j}};
\end{equation}
\begin{equation}
	\phi_{k} = \frac{E_{T,i}\phi_{i} + E_{T,j}\phi_{j}}{E_{T,і} + E_{T,j}};
\end{equation}
\kt-алгоритм є інфрачервоно безпечним та колінеарно безпечним, та, на відміну від конусного алгоритму, він позбавлений неоднозначності при перекриванні струменів~\cite{salam:antikt}.
