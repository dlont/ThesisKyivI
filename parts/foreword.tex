Речі, які зустрічаються у повсякденному житті, на мікроскопічному рівні проявляють регулярні структури такі, як кристалічна ґратка, молекули та атоми. З відкриттям $\alpha,\beta$ та $\gamma$ радіоактивності до рук дослідників потрапили нові природні інструменти для вивчення структур матерії ще менших розмірів.

У 1911 році за допомогою розсіяння $\alpha$-частинок на золоті\\Е.~Резерфорд відкрив існування атомного ядра, розміри якого набагато менші від розміру атома. Композитна структура атомного ядра була встановлена після відкриття Чедвіком нейтрона. Наступними важливими надбаннями були визначення аномального магнітного моменту протона та нейтрона, які відрізнялися від магнітного моменту точкових частинок як, наприклад, електрона. Ці виміри були одними з перших свідчень про подальшу структуру протонів та нейтронів. У 1950-х роках у експериментах Хофштедтера було виявлено специфічний розподіл електричного заряду всередині нуклонів. Логічно постає питання: з чого складаються протон та нейтрон.

У 1964 для пояснення різноманіття „елементарних“ частинок, відкритих на той час, М.~ Гелл-Манном~\cite{GellMann:1964nj} та незалежно Г.~Цвейгом~\cite{Zweig:1964jf} була запропонована ідея кварків. Кварки мали відігравати роль будівельних складових матерії. Наприкінці 1960-х у MIT-SLAC експериментах, які займалися дослідженням глибоко-непружного електрон-протонного розсіяння поза резонансним діапазоном та при набагато вищій віртуальності обмінного фотона, було встановлено наближену незалежність структурної функції $\nu W_2$ від віртуальності фотона (скейлінг), а також те, що поздовжня структурна функція має малу величину. Скейлінгова поведінка була передбачена Бйоркеном на основі методу алгебри струмів~\cite{Bjorken:1968dy}. Рівність нулю поздовжньої структурної функції слідувала з передбачень Каллана-Гросса~\cite{Callan:1969uq} для розсіяння на частинках зі спіном $\nicefrac{1}{2}$. На основі цих спостережень Фейнман запропонував партонну модель~\cite{Feynman:1969ej} точкових ферміонних складових нуклонів, які прямо взаємодіють з обмінним бозоном високої віртуальності у процесі глибоко-непружного розсіяння.

Партонна модель представила новий рівень структур для ферміонів, з яких складаються адрони. Кінцева квантова теорія сильної взаємодії розвивалася протягом наступних років. Ще у 1965 Й. Намбу запропонував $SU\left(3\right)$ калібрувальну теорію Янга-Міллса~\cite{Yang:1954ek} для сильної взаємодії. Ця теорія базувалася на ідеї існування нового тризначного заряду, який потім буде названо кольором. Але на той час ще було не відомо, чи є теорії Янга-Міллса перенормовними. Формалізм, необхідний для коваріантного квантування, був розроблений двома роками пізніше Фаддєєвим та Поповим~\cite{Faddeev:1967fc}. Перенормовність безмасової теорії Янга-Міллса була  доведена т'Хоофтом у 1971~\cite{tHooft:1971fh} і Квантова Хромодинаміка (КХД), як теорія сильної взаємодії, була запропонована Фріцшем та Гелл-Манном у 1972. Партонні складові нуклонів зі спіном $\nicefrac{1}{2}$ були ототожнені з кварками, а безмасові носії кольорової взаємодії --- з глюонами безмасової теорії Янга-Міллса. У 1973, досліджуючи поведінку константи сильної взаємодії Гроссом, Вільчеком та Політцером~\cite{Gross:1973id,Politzer:1973fx} було відкрито асимптотичну свободу.

